%%%%%%%%%%%%%%%%%%%%%%%%%%%%%%%%%%%%%%%%%%%%%%%%%%%%%%%%%%%%%%%%%%%%%%%%%%%%%%%
%                       CARREGA DE LA CLASSE DE DOCUMENT                      %
%                                                                             %
% Les opcions admissibles son:                                                %
%      12pt / 11pt            (cos dels tipus de lletra; no feu servir 10pt)  %
%                                                                             %
% catalan/spanish/english     (llengua principal del treball)                 %
%                                                                             % 
% french/italian/german...    (si necessiteu fer servir alguna altra llengua) %
%                                                                             %
% listoffigures               (El document inclou un Index de figures)        %
% listoftables                (El document inclou un Index de taules)         %
% listofquadres               (El document inclou un Index de quadres)        %
% listofalgorithms            (El document inclou un Index d'algorismes)      %
%                                                                             %
%%%%%%%%%%%%%%%%%%%%%%%%%%%%%%%%%%%%%%%%%%%%%%%%%%%%%%%%%%%%%%%%%%%%%%%%%%%%%%%

\documentclass[11pt,spanish,listoffigures,listoftables]{tfgetsinf}

%%%%%%%%%%%%%%%%%%%%%%%%%%%%%%%%%%%%%%%%%%%%%%%%%%%%%%%%%%%%%%%%%%%%%%%%%%%%%%%
%                     CODIFICACIO DEL FITXER FONT                             %
%                                                                             %
%    windows fa servir normalment 'ansinew'                                   %
%    amb linux es possible que siga 'latin1' o 'latin9'                       %
%    Pero el mes recomanable es fer servir utf8 (unicode 8)                   %
%                                          (si el vostre editor ho permet)    % 
%%%%%%%%%%%%%%%%%%%%%%%%%%%%%%%%%%%%%%%%%%%%%%%%%%%%%%%%%%%%%%%%%%%%%%%%%%%%%%%

\usepackage[utf8]{inputenc} 
%%%%%%%%%%%%%%%%%%%%%%%%%%%%%%%%%%%%%%%%%%%%%%%%%%%%%%%%%%%%%%%%%%%%%
% Para conseguir que la tabla de contenido no salga en rojo
%%%%%%%%%%%%%%%%%%%%%%%%%%%%%%%%%%%%%%%%%%%%%%%%%%%%%%%%%%%%%%%%%%%%%

\hypersetup{ colorlinks=true, linkcolor=black, urlcolor=cyan, }

%%%%%%%%%%%%%%%%%%%%%%%%%%%%%%%%%%%%%%%%%%%%%%%%%%%%%%%%%%%%%%%%%%%%%%%%%%%%%%%
%                        ALTRES PAQUETS I DEFINICIONS                         %
%                                                                             %
% Carregueu aci els paquets que necessiteu i declareu les comandes i entorns  %
%                                          (aquesta seccio pot ser buida)     %
%%%%%%%%%%%%%%%%%%%%%%%%%%%%%%%%%%%%%%%%%%%%%%%%%%%%%%%%%%%%%%%%%%%%%%%%%%%%%%%



%%%%%%%%%%%%%%%%%%%%%%%%%%%%%%%%%%%%%%%%%%%%%%%%%%%%%%%%%%%%%%%%%%%%%%%%%%%%%%%
%                        DADES DEL TREBALL                                    %
%                                                                             %
% titol, alumne, tutor i curs academic                                        %
%%%%%%%%%%%%%%%%%%%%%%%%%%%%%%%%%%%%%%%%%%%%%%%%%%%%%%%%%%%%%%%%%%%%%%%%%%%%%%%

\title{Diseño e implementación de herramientas de análisis de genoma basadas en la teoría de la información}
\author{Cristina Rodríguez Fernández}
\tutor{Jose María Sempere Luna}
\curs{2024-2025}

%%%%%%%%%%%%%%%%%%%%%%%%%%%%%%%%%%%%%%%%%%%%%%%%%%%%%%%%%%%%%%%%%%%%%%%%%%%%%%%
%                     PARAULES CLAU/PALABRAS CLAVE/KEY WORDS                  %
%                                                                             %
% Independentment de la llengua del treball, s'hi han d'incloure              %
% les paraules clau i el resum en els tres idiomes                            %
%%%%%%%%%%%%%%%%%%%%%%%%%%%%%%%%%%%%%%%%%%%%%%%%%%%%%%%%%%%%%%%%%%%%%%%%%%%%%%%

\keywords{retinosi pigmentària, dades genòmiques, teoria de la informació, cadenes de Markov, predicció de mutacions, machine learning, entropia}           % Paraules clau
         {retinosis pigmentaria, datos genómicos, teoría de la información, cadenas de Markov, predicción de mutaciones, machine learning, entropía}        % Palabras clave  
         {retinitis pigmentosa, genomic data, information theory, Markov chains, mutation prediction, machine learning, entropy}                            % Key words

%%%%%%%%%%%%%%%%%%%%%%%%%%%%%%%%%%%%%%%%%%%%%%%%%%%%%%%%%%%%%%%%%%%%%%%%%%%%%%%
%                              INICI DEL DOCUMENT                             %
%%%%%%%%%%%%%%%%%%%%%%%%%%%%%%%%%%%%%%%%%%%%%%%%%%%%%%%%%%%%%%%%%%%%%%%%%%%%%%%

\begin{document}

%%%%%%%%%%%%%%%%%%%%%%%%%%%%%%%%%%%%%%%%%%%%%%%%%%%%%%%%%%%%%%%%%%%%%%%%%%%%%%%
%              RESUMEN DEL TFG EN CASTELLANO, VALENCIA I ANGLES                %
%%%%%%%%%%%%%%%%%%%%%%%%%%%%%%%%%%%%%%%%%%%%%%%%%%%%%%%%%%%%%%%%%%%%%%%%%%%%%%%

\begin{abstract}[spanish]
   La retinosis pigmentaria (RP), una de las distrofias hereditarias de retina (DHR) más frecuente, es conocida por una degradación progresiva en los fotorreceptores que termina por causar pérdida visual irreversible. Aún con los recientes avances en la secuenciación del genoma humano, el origen de esta condición sigue siendo un desafío debido a la amplia diversidad genética involucrada y la cantidad de genes relacionados con las DHR.

   Este Trabajo de Fin de Grado, continuación del Trabajo de Fin de Máster de Andrea Vañó Ribelles y el Trabajo de Fin de Grado de Luis Alberto Martínez Bravo, explora el uso de herramientas de Teoría de la Información y modelos de Fuentes de Markov para analizar el genoma humano, con el objetivo de predecir posibles mutaciones asociadas a la RP. Puesto que la RP es una enfermedad rara y el número de personas afectadas es muy pequeño, las técnicas de Machine Learning no resultan lo suficientemente efectivas debido a la falta de datos para desarrollar modelos robustos.

   En vez de ello, este enfoque se centra en el análisis de regiones genómicas con elevada entropía y densidad de mutaciones, ya que estas áreas serían clave para detectar variantes con mayor probabilidad de estar vinculadas a la enfermedad.

   Para ello, se han empleado los datos genómicos del National Center for Biotechnology Information (NCBI) y archivos VCF proporcionados por el Grupo de Investigación del IIS La Fe de Biomedicina Molecular, Celular y Genómica. El propósito del estudio es profundizar en las características del genoma asociadas a la enfermedad y aportar nuevas maneras de optimizar el diagnóstico genético de la retinosis pigmentaria.
   \\
   \\
\end{abstract}
\begin{abstract}[catalan]
   La retinosi pigmentària (RP) és una de les distròfies hereditàries de retina (DHR) més comunes. Es caracteritza per una degeneració progressiva dels fotoreceptors que provoca una pèrdua de visió irreversible. Malgrat els avanços en la seqüenciació del genoma humà, l'origen d'aquesta malaltia continua sent incert a causa de la gran heterogeneïtat genètica de la patologia i la quantitat de gens associats a les DHR.

   En aquest Treball de Fi de Grau, que amplia el Treball de Fi de Màster d'Andrea Vañó Ribelles i Luis, es proposa la utilització d'eines de la Teoria de la Informació i models de Fonts de Markov per a l'anàlisi del genoma humà amb l'objectiu de predir possibles mutacions associades a la RP. Com que la RP és una malaltia rara i el nombre de persones afectades és reduït, les tècniques de machine learning no resulten prou efectives a causa de l'escassetat de dades per a entrenar models robustos. En lloc d’això, aquest enfocament se centra en l’anàlisi de regions genòmiques amb alta entropia i elevada densitat de mutacions, ja que aquestes zones poden ser clau per a identificar variants amb una major probabilitat d’estar relacionades amb la patologia.

   Per a dur a terme aquest estudi, s'utilitzaran dades genòmiques disponibles al National Center for Biotechnology Information (NCBI) i arxius VCF proporcionats pel Grup d'Investigació de l'IIS La Fe de Biomedicina Molecular, Cel·lular i Genòmica. L'objectiu d'aquest treball és aprofundir en el coneixement de les característiques del genoma associades a la malaltia i aportar noves estratègies que ajuden a optimitzar el diagnòstic genètic de la retinosi pigmentària.
   \\
   \\
\end{abstract}
\begin{abstract}[english]
   Retinitis pigmentosa (RP) is one of the most common hereditary retinal dystrophies (HRD). It is characterized by the progressive degeneration of photoreceptors, leading to irreversible vision loss. Despite advances in human genome sequencing, the origin of this condition remains uncertain due to the high genetic heterogeneity of the disease and the large number of genes associated with HRD.

   In this Bachelor's Thesis, which expands upon the Master's Thesis by Andrea Vañó Ribelles and Luis, the use of Information Theory tools and Markov Source models is proposed for the analysis of the human genome, with the aim of predicting possible mutations associated with RP. Since RP is a rare disease and the number of affected individuals is limited, machine learning techniques are not sufficiently effective due to the lack of data needed to train robust models. Instead, this approach focuses on the analysis of genomic regions with high entropy and a high density of mutations, as these areas may be key to identifying variants with a higher probability of being related to the pathology.

   To achieve this, genomic data available from the National Center for Biotechnology Information (NCBI) and VCF files provided by the Research Group of the IIS La Fe in Molecular, Cellular and Genomic Biomedicine will be used. The purpose of this study is to deepen the understanding of the genomic features associated with the disease and to provide new strategies that help optimize the genetic diagnosis of retinitis pigmentosa.
   \\
   \\
\end{abstract}

%%%%%%%%%%%%%%%%%%%%%%%%%%%%%%%%%%%%%%%%%%%%%%%%%%%%%%%%%%%%%%%%%%%%%%%%%%%%%%%
%                              CONTINGUT DEL TREBALL                          %
%%%%%%%%%%%%%%%%%%%%%%%%%%%%%%%%%%%%%%%%%%%%%%%%%%%%%%%%%%%%%%%%%%%%%%%%%%%%%%%

\mainmatter

%%%%%%%%%%%%%%%%%%%%%%%%%%%%%%%%%%%%%%%%%%%%%%%%%%%%%%%%%%%%%%%%%%%%%%%%%%%%%%%
%                                 INTRODUCCION                                %
%%%%%%%%%%%%%%%%%%%%%%%%%%%%%%%%%%%%%%%%%%%%%%%%%%%%%%%%%%%%%%%%%%%%%%%%%%%%%%%

\chapter{Introducción}

????? ????????????? ????????????? ????????????? ????????????? ?????????????

\section{Motivación}

????? ????????????? ????????????? ????????????? ????????????? ????????????? 

\section{Objetivos}

????? ????????????? ????????????? ????????????? ????????????? ?????????????

\section{Contexto}

????? ????????????? ????????????? ????????????? ????????????? ????????????? 

\section{Estructura de la memoria}

????? ????????????? ????????????? ????????????? ????????????? ????????????? 

%\section{Notes bibliografiques} %%%%% Opcional

%????? ????????????? ????????????? ????????????? ????????????? ?????????????

%%%%%%%%%%%%%%%%%%%%%%%%%%%%%%%%%%%%%%%%%%%%%%%%%%%%%%%%%%%%%%%%%%%%%%%%%%%%%%%
%                                ESTADO DEL ARTE                              %
%%%%%%%%%%%%%%%%%%%%%%%%%%%%%%%%%%%%%%%%%%%%%%%%%%%%%%%%%%%%%%%%%%%%%%%%%%%%%%%

\chapter{Estado del arte}

El campo del análisis genómico de la retinosis pigmentaria (RP) está avanzando rápidamente, con desarrollos en la identificación genética, herramientas de predicción y terapias emergentes. Sin embargo, en un contexto de disponibilidad limitada de especialistas en retina, pruebas y asesoramiento genético, sigue existiendo una gran necesidad de métodos diagnósticos precisos y accesibles. Esta situación ha motivado la búsqueda de nuevas técnicas para mejorar la detección de esta enfermedad.

\section{Métodos tradicionales}

Los métodos tradicionales basados en sistemas de codificación médica como el ICD (International Classification of Diseases) y el SNOMED (Systematized Nomenclature of Medicine) han sido ampliamente utilizados en la medicina para clasificar enfermedades y registrar diagnósticos. Sin embargo, en el caso de enfermedades genéticas raras como la retinosis pigmentaria (RP), estos sistemas de codificación presentan importantes limitaciones. Esto es debido a que es una patología altamente heterogénea a nivel genético y clínico: existen numerosas mutaciones responsables y los síntomas pueden evolucionar de formas muy diversas entre pacientes. 

\section{Objetivos}

????? ????????????? ????????????? ????????????? ????????????? ?????????????

\section{Contexto}

????? ????????????? ????????????? ????????????? ????????????? ????????????? 

\section{Estructura de la memoria}

????? ????????????? ????????????? ????????????? ????????????? ????????????? 

%\section{Notes bibliografiques} %%%%% Opcional

%????? ????????????? ????????????? ????????????? ????????????? ?????????????

%%%%%%%%%%%%%%%%%%%%%%%%%%%%%%%%%%%%%%%%%%%%%%%%%%%%%%%%%%%%%%%%%%%%%%%%%%%%%%%
%                         CAPITOLS (tants com calga)                          %
%%%%%%%%%%%%%%%%%%%%%%%%%%%%%%%%%%%%%%%%%%%%%%%%%%%%%%%%%%%%%%%%%%%%%%%%%%%%%%%

\chapter{Fundamentos teóricos}

????? ????????????? ????????????? ????????????? ????????????? ?????????????

\section{?? ???? ???? ? ?? ??}

????? ????????????? ????????????? ????????????? ????????????? ?????????????

\chapter{Metodología}

????? ????????????? ????????????? ????????????? ????????????? ????????????? 

\section{?? ???? ???? ? ?? ??}

????? ????????????? ????????????? ????????????? ????????????? ?????????????

\chapter{Desarrollo}

????? ????????????? ????????????? ????????????? ????????????? ????????????? 

\section{?? ???? ???? ? ?? ??}

????? ????????????? ????????????? ????????????? ????????????? ?????????????

\chapter{Resultados}

????? ????????????? ????????????? ????????????? ????????????? ????????????? 

\section{?? ???? ???? ? ?? ??}

????? ????????????? ????????????? ????????????? ????????????? ?????????????

%%%%%%%%%%%%%%%%%%%%%%%%%%%%%%%%%%%%%%%%%%%%%%%%%%%%%%%%%%%%%%%%%%%%%%%%%%%%%%%
%                                 CONCLUSIONS                                 %
%%%%%%%%%%%%%%%%%%%%%%%%%%%%%%%%%%%%%%%%%%%%%%%%%%%%%%%%%%%%%%%%%%%%%%%%%%%%%%%

\chapter{Conclusiones}

????? ????????????? ????????????? ????????????? ????????????? ????????????? 

%%%%%%%%%%%%%%%%%%%%%%%%%%%%%%%%%%%%%%%%%%%%%%%%%%%%%%%%%%%%%%%%%%%%%%%%%%%%%%%
%                                BIBLIOGRAFIA                                 %
%%%%%%%%%%%%%%%%%%%%%%%%%%%%%%%%%%%%%%%%%%%%%%%%%%%%%%%%%%%%%%%%%%%%%%%%%%%%%%%

\begin{thebibliography}{10}

%%%%%%%%%%%%%%%%%%%%%%%%%%%%%%%%%%%%%%%%%%%%%%%%%%%%%%%%%%%%%%%%%%%%%%%%%%%%%%%
% MODEL D'ARTICLE                                                             %
%%%%%%%%%%%%%%%%%%%%%%%%%%%%%%%%%%%%%%%%%%%%%%%%%%%%%%%%%%%%%%%%%%%%%%%%%%%%%%%
\bibitem{light}
   Jennifer~S. Light.
   \newblock When computers were women.
   \newblock \textit{Technology and Culture}, 40:3:455--483, juliol, 1999.

%%%%%%%%%%%%%%%%%%%%%%%%%%%%%%%%%%%%%%%%%%%%%%%%%%%%%%%%%%%%%%%%%%%%%%%%%%%%%%%
% MODEL DE LLIBRE                                                             %
%%%%%%%%%%%%%%%%%%%%%%%%%%%%%%%%%%%%%%%%%%%%%%%%%%%%%%%%%%%%%%%%%%%%%%%%%%%%%%%
\bibitem{ifrah}
   Georges Ifrah.
   \newblock \textit{Historia universal de las cifras}.
   \newblock Espasa Calpe, S.A., Madrid, sisena edició, 2008.

%%%%%%%%%%%%%%%%%%%%%%%%%%%%%%%%%%%%%%%%%%%%%%%%%%%%%%%%%%%%%%%%%%%%%%%%%%%%%%%
% MODEL D'URL                                                                 %
%%%%%%%%%%%%%%%%%%%%%%%%%%%%%%%%%%%%%%%%%%%%%%%%%%%%%%%%%%%%%%%%%%%%%%%%%%%%%%%
\bibitem{WAR}
   Comunicat de premsa del Departament de la Guerra, 
   emés el 16 de febrer de 1946. 
   \newblock Consultat a 
   \url{http://americanhistory.si.edu/comphist/pr1.pdf}.

\end{thebibliography}
\cleardoublepage

%%%%%%%%%%%%%%%%%%%%%%%%%%%%%%%%%%%%%%%%%%%%%%%%%%%%%%%%%%%%%%%%%%%%%%%%%%%%%%%
%                           APÈNDIXS  (Si n'hi ha!)                           %
%%%%%%%%%%%%%%%%%%%%%%%%%%%%%%%%%%%%%%%%%%%%%%%%%%%%%%%%%%%%%%%%%%%%%%%%%%%%%%%

\APPENDIX

%%%%%%%%%%%%%%%%%%%%%%%%%%%%%%%%%%%%%%%%%%%%%%%%%%%%%%%%%%%%%%%%%%%%%%%%%%%%%%%
%                         LA CONFIGURACIO DEL SISTEMA                         %
%%%%%%%%%%%%%%%%%%%%%%%%%%%%%%%%%%%%%%%%%%%%%%%%%%%%%%%%%%%%%%%%%%%%%%%%%%%%%%%

\chapter{Configuració del sistema}

????? ????????????? ????????????? ????????????? ????????????? ?????????????

\section{Fase d'inicialització}

????? ????????????? ????????????? ????????????? ????????????? ?????????????

\section{Identificació de dispositius}

????? ????????????? ????????????? ????????????? ????????????? ?????????????

%%%%%%%%%%%%%%%%%%%%%%%%%%%%%%%%%%%%%%%%%%%%%%%%%%%%%%%%%%%%%%%%%%%%%%%%%%%%%%%
%                               ALTRES  APÈNDIXS                              %
%%%%%%%%%%%%%%%%%%%%%%%%%%%%%%%%%%%%%%%%%%%%%%%%%%%%%%%%%%%%%%%%%%%%%%%%%%%%%%%


\chapter{??? ???????????? ????}

????? ????????????? ????????????? ????????????? ????????????? ????????????? 



%%%%%%%%%%%%%%%%%%%%%%%%%%%%%%%%%%%%%%%%%%%%%%%%%%%%%%%%%%%%%%%%%%%%%%%%%%%%%%%
%                              FI DEL DOCUMENT                                %
%%%%%%%%%%%%%%%%%%%%%%%%%%%%%%%%%%%%%%%%%%%%%%%%%%%%%%%%%%%%%%%%%%%%%%%%%%%%%%%

\end{document}
